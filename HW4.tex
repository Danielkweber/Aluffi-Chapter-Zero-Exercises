\documentclass{article}

\usepackage{amsmath}
\usepackage{amssymb}
\usepackage{amsthm}
\usepackage{enumitem}
\usepackage[margin = 1in]{geometry}
\usepackage[mathscr]{euscript}

\title{Honors Algebra HW4}
\author{Daniel Weber dweber11}

\newcommand{\Solution}{\textit{Solution: }}
\newcommand{\R}{$\mathbb{R}$}


\begin{document}
    \maketitle
    \begin{enumerate}
        \item[\textbf{Problem 7.7}]
            \begin{quote}
                \Solution We begin by noting that to prove a subgroup $N$ is normal, we must show that $gNg^{-1} = N$. In
                our example, every element on subgroup $N$ is of order $n$ (i.e. $h \in N, h^n = e$). 
                Therefore, we do the following calculation.
                \begin{equation*}
                    \begin{gathered}
                        (gNg^{-1})^n = N^n \\
                        gN^ng^{-1} = e \\
                        geg^{-1} = e \\
                        e = e
                    \end{gathered}
                \end{equation*}
                Since we have show that $gNg^{-1} = N$, our subgroup is normal. \qedsymbol
            \end{quote} 
        \item[\textbf{Problem 7.11}]
            \begin{quote}
                \Solution Consider $H = [G, G]$. Note that every element of $H$ can be represented in the form $aba^{-1}b^{-1}$.
                Now, consider the $h \in H, ghg^{-1}$.
                \begin{equation*}
                    \begin{gathered}
                        ghg^{-1} = ghg^{-1}e \\
                        ghg^{-1} = ghg^{-1}(h^{-1}h) \\
                        ghg^{-1} = (ghg^{-1}h^{-1})h
                    \end{gathered}
                \end{equation*}
                Now, note that $ghg^{-1}h^{-1}$ is in the form $aba^{-1}b^{-1}$ and is hence part of the subgroup $[G, G]$.
                Also, note that a subgroup is closed so multiplication of two elements in it results in another element of 
                the subgroup. Therefore, $ghg^{-1} = (ghg^{-1}h^{-1})h \in H$, thus proving that $[G, G]$ is normal. Next,
                we prove commutativity of $[G, G]$. This means that we are trying to show that $\forall x, y \in G, \;
                xyH = yxH \therefore x^{-1}y^{-1}xyH = H$. Now, note that $x^{-1}y^{-1}xy$ is in 
                $[G, G] = H \implies x^{-1}y^{-1}xyH = H$. Therefore, the result is proven. \qedsymbol
            \end{quote}
        \item[\textbf{Problem 7.13}]
            \begin{quote}
                We begin by noting that the result of exercise 7.12 allows shows that $F(A) / [F(A), F(A)] \backsimeq F^{ab}(A)$. 
                Next we note that since taking the quotient group with the commutator subgroup simply adds an abelian structure to 
                the group, doing it on both sides of our given isomorphism preserves the isomorphism. Therefore, 
                $F^{ab}(A) \backsimeq F(A) / [F(A), F(A)] \backsimeq F(B) / [F(B), F(B)] \backsimeq F^{ab}(B)$. Next, we note that
                Proposition 5.6 in the book tells us that $F^{ab}(A) \backsimeq \mathbb{Z}^{\oplus A}$. Therefore, 
                $\mathbb{Z}^{\oplus A} \backsimeq F^{ab}(A) \backsimeq F^{ab}(B) \backsimeq \mathbb{Z}^{\oplus B}$. In this form, it is clear to see that if $A$
                is finite, then $B$ must be finite as well. The reason for this is because the above isomorphism would not hold if one of
                these sets were infinite and the other were not. \qedsymbol
            \end{quote}
        \item[\textbf{Problem 7.14}]
            \begin{quote}
                \Solution We want to show that for $f \in Aut(G), \: f \circ \gamma_g \circ f^{-1} \in Inn(G)$.
                \begin{equation*}
                    \begin{gathered}
                        f \circ \gamma_g \circ f^{-1} (a) = f(gf^{-1}(a)g^{-1}) \\
                        = f(g)f(f^{-1}(a))f(g^{-1}) \\
                        = f(g)af(g^{-1}) \\
                        = f(g)af(g)^{-1}
                    \end{gathered}
                \end{equation*}
                Which is the form of an inner automorphism. Therefore, $Inn(G)$ is normal. \qedsymbol
            \end{quote}
        \item[\textbf{Problem 8.7}]
        \begin{quote}
            \Solution In order to prove this result, we first construct the inclusion functions into the 
            coproduct using the universal property of quotients. We then define the coproduct homomorphism
            element-wise to show that our presentation satisfies the coproduct. First, consider the functions
            $i_A: (A \vert \mathcal{R}) \to (A \cup A' \vert \mathcal{R} \cup \mathcal{R'})$ and 
            $i_{A'}: (A' \vert \mathcal{R'}) \to (A \cup A' \vert \mathcal{R} \cup \mathcal{R'})$. We can construct these 
            functions using the universal property of quotients. WLOG, we will consider the set $F(A)$. Now, consider a 
            function $f: F(A) \to (A \cup A' \vert \mathcal{R} \cup \mathcal{R'})$. Clearly, $\mathcal{R} \subseteq ker f$.
            Therefore, the universal property of quotients guarantees that there exists a group homomorphism 
            $\varphi: F(A) / \mathcal{R} \to (A \cup A' \vert \mathcal{R} \cup \mathcal{R'})$. The same is true for the 
            $A', R'$ presentation. Next, we construct a morphism $\alpha: (A \cup A' \vert \mathcal{R} \cup \mathcal{R'}) 
            \to A$ for an arbitrary object $P$ which in order to make the coproduct diagram commute. We define this function
            element-wise; for $g_A: (A \vert \mathcal{R}) \to P$ and $g_{A'}: (A' \vert \mathcal{R'}) \to P$ we define $\alpha$
            as such
            \begin{gather*}
                \begin{cases}
                    \alpha(i_A(a)) = g_A(a), \: a \in (A \vert \mathcal{R}) \\
                    \alpha(i_{A'}(a')) = g_{A'}(a'), \: a' \in (A' \vert \mathcal{R'})
                \end{cases}
            \end{gather*}
            This makes the coproduct diagram commute and we have thus proven that $(A \cup A' \vert \mathcal{R} \cup \mathcal{R'})$
            satisfies the universal property of coproducts. \qedsymbol
        \end{quote} 
        \item[\textbf{Problem 8.21}]
            \begin{quote}
                We begin by defining a function $f: k \mapsto kH$ and noting that the $ker f = H \cap K$. This means that 
                $K/K \cap H \backsimeq KH \backsimeq \{kH \vert k \in K\}$ by the universal property of quotients. 
                Next, we also note that $HK/H \backsimeq \{kH \vert k \in K\}$. Therefore, $HK/H \backsimeq K/K \cap H$.
                We now prove the formula shown in the problem. First, it is obvious that the index of our two above
                isomorphic groups are of equivalent index. (Firstly, finite isomorphic groups must be of the same size. 
                Secondly, we have shown that they are both isomorphic to ${kH \vert k \in K}$). Namely, 
                \begin{equation*}
                    \vert\frac{HK}{H}\vert = [HK : H] = |\frac{K}{K \cap H}|
                \end{equation*}
                which we have shown to be true above (by $\{kH \vert k \in K\}$). By Lagrange's Theorem, 
                \begin{equation*}
                    \begin{gathered}
                        \frac{|HK|}{|H|} = [HK : H] = \frac{|K|}{|K \cap H|} \\
                        \implies \frac{|HK|}{|H|} = \frac{|K|}{|K \cap H|} \\
                        |HK| =  \frac{|H||K|}{|K \cap H|} \quad\qedsymbol
                    \end{gathered}
                \end{equation*}

            \end{quote} 
        \item[\textbf{Problem 9.10}]
            \begin{quote}
                I claim that the function $f: G/H \to H \backslash G$ such that $gH \mapsto (gH)^{-1}$ is a bijective mapping
                between the set of left cosets and the set of right cosets. First, we verify that $(gH)^{-1} = H^{-1}g^{-1}$ 
                is indeed a right coset of $G/H$. First, we realize that $H^{-1} = H$ because $H$ is a subgroup. Therefore, if 
                we take the inverse of every element, we will just arrive back at $H$. Next, we note that $g^{-1}$ is an element
                of $G$. Therefore, $H^{-1}g^{-1}$ is a right coset. To prove that this map is bijective, we just show that it is 
                invertible. The function $\beta: (gH)^{-1} \mapsto gH$ is trivially the inverse of $f$. Therefore, there is a 
                bijective mapping between the right and left cosets of $G/H$. \qedsymbol
            \end{quote} 
        \item[\textbf{Problem 9.12}]
            \begin{quote}
                We prove this by the template provided in exercise 9.11. First, we interpret the action of $G$ on $G/H$ as a 
                homomorphism $\sigma: G \to S_n$. Next, we show that 
                the $ker\sigma \subseteq H$. Consider the action $x \cdot gH$ for $x \in G$. The elements of the kernel are the
                elements $x$ s.t. $xgH = gH$. In particular, this must be true when $gH = H$. Therefore, consider when $xH = H$,
                this obviously occurs when $x \in H$ which shows that the $ker\sigma \subseteq H$. Since every kernel is normal,
                we have found a normal subgroup $K \subseteq H$. Next, we note that $G/K$ ($K = ker\sigma$) is isomorphic to a subgroup
                of $S_n$. Then, by Lagrange's Theorem, $[G : K]$ must have order dividing $n!$. In addition, because $K$ is a subgroup 
                of $G$, $[G : K]$ must also divide $|G|$. \qedsymbol 
            \end{quote}
        \item[\textbf{Problem 9.15}]
            \begin{quote}
                We begin by noting that for a group action to be called "free", only the identity of the G can fix any element 
                of the set being acted upon (i.e. if $ga = a$ then $g = e_G$). Next, we note that a group action on a Cayley graph
                acts on the vertices of that graph. Now, we must prove that a group $G$ acts freely on its corresponding 
                Cayley graph. To prove that the action is free, we must only realize that for any element $g \in G$ if $hg = g$ then 
                $h = e_G$. Next, we must prove that this 
                is a valid action. We do this by showing that this action preserves incidence in the graph. If two vertices 
                $g_1, g_2 \in G$ are connected in the Cayley graph, then $\exists a$ such that $g_2 = g_1a$. Since we are 
                considering a left action, the action obviously preserves incidence. Let $h \in G$, we assume that $g_2 = g_1a$
                we want to show that $hg_2 = hg_1a$.
                \begin{equation*}
                    hg_2 = hg_1a \; \therefore \; h^{-1}hg_2 = h^{-1}hg_1a \; \therefore \; g_2 = g_1a
                \end{equation*}
                Thus showing that our action preserves incidence. 
                
                Next, we show that the corresponding Cayley graph of a free group is a tree. Assume by contradiction that there 
                exists a cycle in the Cayley graph of a free group. This means that for an element $b \in F(A)$ we can multiply by 
                the generators of $F(A)$ a finite number of times and return to $b$. Now, consider $bg_ig_i...g_i = b$ where the 
                $g_i$ represent generators of $F(A)$. Now, we left multiply by $b^{-1}$ to get $g_ig_i...g_i = e$. However, 
                we have no guarantee that this is the case in a free group (in fact, it is typically not the case). Therefore, 
                we conclude that the Cayley graph of a free group must be acyclic and hence a tree. \qedsymbol
            \end{quote} 
        \item[\textbf{Problem 9.16}]
            \begin{quote}
                Consider the action of a free group $F(A)$ on a tree $\Gamma$ denoted by $\rho:F(A) \times \Gamma \to \Gamma$.
                Now, consider a $S \subseteq F(A)$ which is a subgroup of $F(A)$. Note that the group action
                $\rho_S:S \times \Gamma \to \Gamma$ obviously also acts freely on $\Gamma$ because it inherits the identity 
                element from $F(A)$ and we have already assumed that only the identity element of $F(A)$ fixes any element of
                $\Gamma$. Further, since $S \subseteq F(A)$ there are obviously no elements of $S$ which fix an element of $\Gamma$. 
                Therefore, since a subgroup of $F(A)$ can act freely on a tree and every free group acts freely on a tree,
                every subgroup of a free group must be free. \qedsymbol
            \end{quote}
    \end{enumerate}
\end{document}